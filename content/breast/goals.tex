
\graphicspath{{./content/breast/figures/}}

\begin{frame}\frametitle{Goals}
  \begin{figure}
  \begin{adjustbox}
    {max size={.95\textwidth}{.33\textheight},keepaspectratio=true}
    \begin{tabular}[t]{cc}
      \includegraphics[trim=30 130 50 20,clip,height=.40\textheight,width=.45\textwidth]{a110105_094.png} &
      \includegraphics[trim=20 75 30 10,clip,height=.40\textheight,width=.45\textwidth]{segment.png}\\
      \includegraphics[trim=50 60 15 0,clip,width=.45\textwidth]{000023.png} &
      \includegraphics[trim=50 60 15 0,clip,width=.45\textwidth]{000023_gt.png}
    \end{tabular}%
  \end{adjustbox}
  \caption{Lestion and tissue delineation}
\end{figure}
\end{frame}


\frame{\frametitle{Breast \ac{us} tissue labeling}%\framesubtitle{\insertsubsection}

  \begin{figure}
    \begin{tikzpicture}
      %[every node/.style={draw,line width=2pt,inner sep=0,outer sep=0}]
      \draw node  () {
        \begin{adjustbox}
          {max size={.95\textwidth}{.95\textheight},keepaspectratio=true}
          \input{./content/breast/framework.tex}
        \end{adjustbox}
      }(0,0);
    \end{tikzpicture}
    \caption{Strategy used for segmenting breast tissues and lesions in \ac{us} data.}
  \end{figure}
}
